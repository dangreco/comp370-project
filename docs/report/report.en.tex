%File: formatting-instruction.tex
\documentclass[letterpaper]{article}
\usepackage{aaai24}
\usepackage{times}
\usepackage{helvet}
\usepackage{courier}
\usepackage{amsmath}
\usepackage{amssymb}
\usepackage{graphicx}
\usepackage{url}
\usepackage[backend=biber]{biblatex}
\addbibresource{refs.bib}
\frenchspacing
\setlength{\pdfpagewidth}{8.5in}
\setlength{\pdfpageheight}{11in}
\pdfinfo{
  /Title (COMP 370 Final Project)
/Author (Daniel Greco, Denis Tsariov, Kejun Fang)}
\setcounter{secnumdepth}{0}
\begin{document}
% The file aaai.sty is the style file for AAAI Press
% proceedings, working notes, and technical reports.
%
\title{Speaking of Seinfeld: \\ Dialogue Analysis of Side Characters
in \textit{Seinfeld}}
\author{Daniel Greco, Denis Tsariov, Kejun Fang\thanks{Code and data
available at: \protect\url{https://github.com/dangreco/comp370-project}}}
\maketitle

% ===============================
% INTRODUCTION
% ===============================
\section{Introduction}

\par
Television sitcoms offer a rich corpus for studying character
development and dialogue patterns. Side characters, though appearing
less frequently than protagonists, often serve crucial narrative
functions and exhibit distinctive speech patterns that contribute to
a show's comedic identity. Understanding how these characters
communicate and what topics they discuss can provide insights into
both character design and audience engagement.

\subsection{Research Questions}

\par
In this study, we analyze the dialogue of side characters in the NBC
sitcom \textit{Seinfeld} (1989-1998), a show renowned for its observational
humor and ensemble cast. Specifically, we investigate two research
questions:

\begin{itemize}
  \item \textbf{RQ1}: What topics do side characters tend to talk about?
  \item \textbf{RQ2}: How much attention does each character give to
    each topic?
\end{itemize}

\noindent
By answering these questions, we aim to characterize the thematic focus
of individual side characters and identify patterns in their dialogue.

\subsection{Approach}

\par
We collected scripts from 176 episodes and focused our analysis on
five prominent side characters: Frank Costanza, Helen Seinfeld, Morty
Seinfeld, Newman, and Susan Ross. Through open coding, we developed
an eight-category typology for classifying dialogue topics, then
manually annotated over 2,000 lines of dialogue. We supplemented
human annotation with LLM-based classification to resolve
disagreements and validate our results. Our analysis reveals distinct
topical profiles for each character, reflecting their narrative roles
and relationships with the main cast.

\subsection{Key Findings}

\par
By quantifying topic engagement, our analysis revealed three core findings:

\subsubsection{1.}
\par
The results confirm thematic specialization as character definition.
Morty Seinfeld embodies the ``provider/hustler'' archetype, exhibiting
a unique specialization in Money, which constitutes nearly 15\% of
his lines; Frank Costanza functions as an ``agent of chaos,'' and his
dialogue is co-dominated by Emotion/Feeling (24.3\%) and Food,
reflecting his high volatility. Furthermore, Newman is the only
character whose most dominant topics include both Food and Work,
which is consistent with his role as a mailman and frequent snacker.

\subsubsection{2.}
\par
Our analysis distinguishes narrative roles: drivers vs. reactors.
Susan Ross's dialogue is heavily weighted toward Relationships, aimed
at advancing the romantic (and later tragic) plotline; conversely,
Helen Seinfeld acts as a ``narrative stabilizer,'' whose thematic
distribution is more balanced, lacking sharp topical spikes, and
primarily reacting to the obsessions of others.

\subsubsection{3.}
\par
TF-IDF analysis emphasizes the self-referential nature of ``Culture''.
Culture and Health topics are not anchored in general pop culture but
are deeply rooted in specific \textit{Seinfeld} show-lore, such as
``Festivus`` and Frank Costanza’s ''Serenity Now'' mantra, highlighting
the insular nature of the show's universe.

\subsection{Report Structure}
\par
The remainder of this report is structured as follows. Section 2
systematically introduces the dataset construction process, including
data collection methods and associated design decisions. Section 3
presents the methods, including topic system construction, data
annotation procedures, and topic characterization analysis. Section 4
presents all results, sharing all findings, including the topics
selected with their definitions, topic characterization, and topic
engagement analysis. Section 5 is the Discussion, interpreting the
significance and implications of the results.

% ===============================
% DATA
% ===============================
\section{Data}

\par
Our dataset was constructed by scraping episode scripts from the
Internet Movie Script Database (IMSDb) \cite{imsdb} and character
metadata from the Seinfeld Fandom Wiki \cite{seinfeldwiki}. The result of the
scraping pipeline is 176 \textit{Seinfeld} episodes across all nine
seasons, with structured dialogue data including speaker
identification, line numbers, and the dialogue text itself. Two clip
show episodes were excluded from our dataset (the 100th Episode
Special and The Clip Show, Parts I \& II), as their scripts were
unavailable on IMSDB. Character information, including names, gender,
occupation, and actor portrayals, was gathered by crawling character
pages alphabetically from the Fandom Wiki, for a total of 330 unique
characters. Data is stored in a SQLite \cite{sqlite} database using
SQLAlchemy ORM \cite{sqlalchemy};
HTTP response caching is implemented to minimize repeated requests
against the source websites during development and testing. Rate
limits, crawl delays, and prohibited resources for crawling were
respected where applicable for each site.

\subsection{Name Resolution}

\par
Due to the nature of scripts from the IMSDb being contributor-submitted,
there were often formatting inconsistencies with character cues. To
accurately attribute lines to characters, we devised a name
resolution algorithm. The algorithm calculates the maximum likelihood
estimation (MLE) of character identity by Jaro-Winkler distance
\cite{jellyfish}, phonetic similarity \cite{jellyfish}, and character popularity
metrics to resolve ambiguous or variant character names to their
canonical forms. The MLE estimate for an observed name $n_{obs}$ is given by:

\small
\[
  \hat{c} = \arg \max_{c_i \in C} \left[ \Pr(n_{obs} | c_i),
  \mathrm{Popularity}(c_i) \right]
\]
\normalsize

\noindent
...where the maximization is lexicographic (prioritizing likelihood,
then popularity), and:

\small
\[
  \Pr(n_{obs} | c_i) = \min \left( 1.0,
    \mathrm{JW}(n_{obs}, n_i) + 0.1 \cdot \mathrm{Ph}(n_{obs}, n_i)
  \right)
\]
\normalsize

\noindent
...with $\mathrm{Popularity}(c_i)$ denoting the popularity rank of
character $c_i$ (higher is more popular), $\mathrm{JW}(a, b)$
denoting the Jaro-Winkler distance between strings $a$ and $b$, and
$Ph(a, b)$ equating to 1 if strings $a$ and $b$ are a phonetic match
(via Metaphone comparison), 0 otherwise. The popularity of a
character $\mathrm{Popularity}(c_i)$ is given by:

\small
\[
  \mathrm{Popularity}(c_i) = \frac{\deg(c_i)}{\sum_{c \in C} \deg(c)}
\]
\normalsize

\noindent
...where the in-degree of a character $c_i$, $\deg(c_i)$, is given by:

\small
\[
  \deg(c_i) = \sum_{c \in C, c \neq c_i} \left| \{ a \mid a
  \text{ points to $c_i$}, \forall a \in \mathrm{anchors}(c)  \} \right|
\]
\normalsize

\noindent
...where $\mathrm{anchors}(c)$ is the set of all \texttt{a}
(anchor) tags on the Fandom character page for character $c$.

\subsection{Character Selection}

\begin{table}[htbp]
  \centering
  \begin{tabular}{|l|c|}
    \hline
    \textbf{Character Type} & \textbf{Appearance Frequency} \\ \hline
    Main & $f(c) \geq 0.70$ \\ \hline
    Side & $0.10 \leq f(c) < 0.70$ \\ \hline
    Recurring & $0.05 \leq f(c) < 0.10$ \\ \hline
    Guest & $f(c) < 0.05$ \\ \hline
  \end{tabular}
  \caption{
    Character typing based on episode appearance frequency.
    Thresholds experimentally chosen to accurately reflect
    narrative importance.
  }
  \label{fig:character_typing}
\end{table}

\par
The complete corpus counts a total of 46,179 dialogue lines, from
which 39,668 are distinct. In order to recognize side characters for
the analysis, we categorized characters based on their episode
appearance frequency. We experimentally iterated on frequency bounds
in order to accurately capture the notion of character type based on
our prior knowledge of the show. We eventually reached stability with
the parameters listed in Table \ref{fig:character_typing}. From the side
character category, we chose the top five characters by the number of
dialogue lines: Frank Costanza, Helen Seinfeld, Morty Seinfeld,
Newman, and Susan Ross.

\subsection{Line Selection}

\par
For each selected character, we extracted dialogue lines with a
minimum length of 15 characters to filter out trivial utterances such
as greetings or single-word responses. The filtering step was
necessary so that our analysis focuses on substantive speech acts
rather than conversational filler. After applying the filter, we
collected 2,150 lines across the five side characters:

\begin{table}[htbp]
  \centering
  \begin{tabular}{|l|c|}
    \hline
    \textbf{Character} & \textbf{Line Count} \\ \hline
    Newman & 547 \\ \hline
    Morty Seinfeld & 474 \\ \hline
    Helen Seinfeld & 421 \\ \hline
    Frank Costanza & 393 \\ \hline
    Susan Ross & 315 \\ \hline
  \end{tabular}
  \caption{Post-processing line counts per character.}
  \label{fig:character_lines}
\end{table}

\noindent
...giving an average of 430 lines per character. This quantity is
above the 300
lines required per character for the analysis, considering that there
will be a lot of banter. It also gives enough data for both the open
coding phase (100 lines per character) and the full annotation of
non-banter dialogue. We randomly sampled lines from the filtered set,
taking care not to bias our sample toward particular topics or
narrative contexts. For reproducibility, this random sampling is done
with a fixed random seed.

% ===============================
% METHODS
% ===============================
\section{Methods}

\par
Our methodology consisted of three phases: codebook development
through open coding, manual annotation, and topic characterization.

\subsection{Codebook Development}

\par
To develop our topic categories, each team member independently
conducted open coding on 100 randomly sampled lines from each of the
five selected characters (500 lines total per annotator). Each line
was assigned to exactly one topic category. After independent coding,
the team compared their resulting typologies and discussed
discrepancies. We converged quickly on a final codebook containing
eight categories: Food, Relationships, Work, Money, Lifestyle,
Culture, Health, and Miscellaneous. The Miscellaneous category
captures lines that do not fit into any specific topic, including
greetings and uncategorizable content. Each category in the codebook
includes a description and annotated examples indicating whether a
given line should be included and why.

\subsection{Annotation Process}

\par
We used Label Studio as our primary annotation interface for manual
labeling of the full dataset. Each team member independently
annotated the complete set of 2,150 lines. Annotators were presented
with individual dialogue lines without surrounding context, requiring
them to classify each line based solely on its content. This design
decision ensured that topic assignments reflected the intrinsic
content of each utterance rather than being influenced by
conversational flow. In addition to manual annotation, we employed
two large language models (\texttt{gpt-oss:120b} \cite{gptoss120} and
\texttt{minimax-m2} \cite{minimaxm2}) to annotate the
complete dataset.

\begin{table}[htbp]
  \centering
  \begin{tabular}{|l|c|c|}
    \hline
    \textbf{Annotator} & \textbf{Type} & \textbf{Weight} \\ \hline
    Daniel Greco & Human & 1.0 \\ \hline
    Denis Tsariov & Human & 1.0 \\ \hline
    Kejun Fang & Human & 1.0 \\ \hline
    \texttt{gpt-oss:120b} & LLM & 0.75 \\ \hline
    \texttt{minimax-m2} & LLM & 0.5 \\ \hline
  \end{tabular}
  \caption{
    Annotator weights in a majority voting scheme.
    Note that LLMs are only used as tie-breakers, hence their lower weights.
  }
  \label{fig:annotator_weights}
\end{table}

\par
To resolve disagreements, we used a majority voting scheme among the
three human annotators. The algorithm we specifically used was
\textit{hierarchical weighted majority voting} \cite{meyen2021group}.
When a majority could not achieve consensus on a category, the LLM
annotations were factored in as tie-breakers. If the LLMs also
disagreed on a category suggested by the human annotators, the line
was marked as \texttt{UNKNOWN}, indicating that the content was too
ambiguous to accurately label (i.e. banter that slipped through our
initial filter.) This process resulted in only 8 lines marked as
\texttt{UNKNOWN} out of 2,150 total ($\approx 0.37$\%), demonstrating
strong overall agreement.

\subsection{Topic Characterization}

\par
To identify the most representative terms for each topic category, we
applied term frequency-inverse document frequency (TF-IDF) weighting.
For each topic, we aggregated all annotated lines belonging to that
category into a single document, then computed TF-IDF scores across
the resulting corpus of eight topic documents. We additionally
calculated a distinctiveness score for each term, measuring how
uniquely associated a word is with its highest-scoring topic relative
to other topics. Terms with high TF-IDF scores and high
distinctiveness represent vocabulary that both frequently appears
within a topic and rarely appears elsewhere, providing interpretable
characterizations of each topical category.

% ===============================
% RESULTS
% ===============================
\section{Results}

\subsection{Topics Selected}

\par
We identified nine topical categories representing the primary
conversational domains of \textit{Seinfeld}'s side characters. Definitions for
each category are provided below.

\begin{itemize}
  \item \textbf{Food}: Lines about eating, food, restaurants, meals,
    and dining experiences.
  \item \textbf{Relationships}: Lines about \textit{romantic} relationships,
    dating, attraction, breakups, and interpersonal dynamics within
    romantic contexts.
  \item \textbf{Work}: Lines about jobs, workplace responsibilities,
    supervisors, coworkers, or other career-related matters.
  \item \textbf{Money}: Lines about money, prices, costs, deals, or
    economic transactions.
  \item \textbf{Lifestyle}: Lines about apartments, living
    arrangements, neighborhoods, city locations, and daily life in
    New York City.
  \item \textbf{Culture}: Lines about sports, TV shows, movies,
    celebrities, or other popular culture references.
  \item \textbf{Health}: Lines about medical issues, physical
    conditions, illnesses, bodily discomfort, or interactions with doctors.
  \item \textbf{Emotion/Feeling}: Lines about emotional reactions,
    subjective states, strong opinions, or other emotion-driven
    statements. Note that expressions of romantic attraction are
    categorized under \emph{Relationships}.
  \item \textbf{Miscellaneous}: Lines about general statements,
    greetings, or other content that does not fit any specific topic category.
\end{itemize}

\subsection{Topic Characterization}

\begin{table}[h]
  \centering
  \scriptsize
  \begin{tabular}{|l|c|c|c|c|c|}
    \hline
    \textbf{Topic} & \textbf{$t_1$} & \textbf{$t_2$} & \textbf{$t_3$} &
    \textbf{$t_4$} & \textbf{$t_5$} \\ \hline

    Culture & festivus & movie & watch & guide & danson \\ \hline
    Emotion & good & like & thank & kid & sorry \\ \hline
    Food & eat & glass & dinner & cook & pie \\ \hline
    Health & doctor & problem & sleep & man & tape \\ \hline
    Lifestyle & stay & place & house & sleep & apartment \\ \hline
    Money & pay & dollar & money & wallet & cent \\ \hline
    Relationships & love & woman & friend & wedding & beautiful \\ \hline
    Work & work & banker & year & business & desk \\ \hline
  \end{tabular}
  \normalsize
  \caption{Top 5 terms per topic ranked by TF-IDF score.}
  \label{tab:tfidf}
\end{table}

\par
To validate our topic definitions and understand the lexical markers
of each category, we analyzed the top terms identified by our TF-IDF
scoring. The five highest-scoring terms for each category are shown
in Table 1. Analysis here shows that the speech of side characters is
bimodal -- either highly ground in the lore of \textit{Seinfeld} or
representative of linguistic patterns.

\par
For example, the \emph{Culture}
category is highly determined by ``festivus'' ($w \approx 0.06$)
and ``danson''
($w \approx 0.03$). The former, ``festivus'', refers to a secular and
non-commmercial alternative holiday to Christmas celebrated by the
Costanzas \cite{seinfeldwiki}. The latter, ``danson'', refers to
actor Ted Danson, with
whom George Costanza has an ongoing obsession with throughout the
series \cite{seinfeldwiki}.

\par
Likewise, the \emph{Health} category includes ``serenity'' ($w \approx
0.02$), which links health discussions to Frank Costanza's
blood-pressure mantra ``serenity now'' -- a line often repeated by
Frank Costanza, specifically in the the episode ``The Serenity Now''
\cite{seinfeldwiki}.

\par
The \emph{Lifestyle} category captures the specific geography of the
show, with ``del boca vista'' ($w \approx 0.02$) and ``florida'' ($w
\approx 0.02$) referring to the fictional retirement community within
\textit{Seinfeld} \cite{seinfeldwiki}, whereas ``apartment'' and
``street'' refer to more general aspects of the locale of the show,
New York City. \emph{Money} is characterized by transactional terms
(``pay,'' ``dollar,'' ``wallet''), while \emph{Work} reflects the
white-collar and service roles of the characters (``banker,''
``deliver,'' ``interview'').

\subsection{Topic Engagement by Character}

\par
We analyzed the distribution of topics for each character to answer
our second research question regarding character attention. The
aggregated topic distributions reveal distinct archetypes for
each side character.

\begin{figure}[htbp]
  \centering
  \includegraphics[width=\linewidth]{fig/topic_proportions_by_character.png}
  \caption{Proportion of dialogue lines dedicated to each topic per
    character. Note Morty Seinfeld's distinct focus on Money and Susan
  Ross's focus on Relationships.}
  \label{fig:topic_distribution}
\end{figure}

\par
\textbf{Frank Costanza} is defined by high volatility; his dialogue
is dominated by \emph{Emotion/Feeling} (24.3\% of lines) and
\emph{Food}, reflecting his tendency toward loud outbursts and family
dinner scenes.
\textbf{Morty Seinfeld} shows a unique specialization in
\emph{Money}, which constitutes nearly 15\% of his
dialogue—significantly higher than any other character—aligning with
his obsession over ``the deal'' and financial details.
\textbf{Susan Ross}, as George's fiancé, is heavily skewed toward
\emph{Relationships} and \emph{Emotion}, which together make up over
35\% of her lines.
\textbf{Newman} exhibits the highest engagement with \emph{Food}
among all subjects, alongside a substantial \emph{Work} component,
consistent with his role as a mailman and frequent snacker.
\textbf{Helen Seinfeld} acts as a narrative stabilizer; her
distribution is the most balanced, with moderate engagement across
\emph{Relationships}, \emph{Emotion}, and \emph{Lifestyle}, often
reacting to the obsessions of those around her rather than driving a
single thematic niche.

\subsubsection{Evolution Across Seasons}

\par
Longitudinal analysis highlights how character roles shifted over time.
\textbf{Morty Seinfeld} sees a dramatic spike in \emph{Money}-related
dialogue in Season 6, where it peaks at 40\% of his lines, coinciding
with storylines focused on his wallet and business ventures.
\textbf{Susan Ross} displays a clear trajectory defined by her
relationship with George. Her dialogue is consistently dominated by
\emph{Relationships}, but shifts toward \emph{Emotion/Feeling} in
Season 7 leading up to her death. The apparent spike in Season 9
(40\% \emph{Emotion}) is an artifact of sparse data; Susan appears in
only 5 lines in the finale (flashbacks), 2 of which were emotional in nature.
\textbf{Newman} displays a notable anomaly in Season 2, where
\emph{Culture} appears to constitute over 30\% of his dialogue. This
is due to an extremely small sample size (only 3 lines total in
Season 2), where a single reference to a cultural topic skewed the
proportion. In later seasons (3--9), his profile stabilizes with
consistent contributions to \emph{Food} and \emph{Work}.

\begin{figure}[htbp]
  \centering
  \includegraphics[width=\linewidth]{fig/topic_season_evolution_morty.png}
  \caption{Morty Seinfeld's topic evolution, highlighting the
  Season 6 spike in 'Money'.}
  \label{fig:morty_evolution}
\end{figure}

\begin{figure}[htbp]
  \centering
  \includegraphics[width=\linewidth]{fig/topic_season_evolution_susan.png}
  \caption{Susan Ross's topic evolution. Note the shift to
  'Emotion' in Season 9 (flashback scenes).}
  \label{fig:susan_evolution}
\end{figure}

% ===============================
% DISCUSSION
% ===============================
\section{Discussion}

\par
Our analysis suggests that the dialogue of \textit{Seinfeld} side
characters is not merely conversational filler but is highly
specialized in order to reinforce distinct character archetypes. By
quantifying engagement with topics, we can map the narrative
functions these characters serve.

\subsection{Thematic Specialization as Character Definition}

\par
The results indicate that side characters in \textit{Seinfeld} are
often defined by a single, dominating obsession that anchors their
comedic role.

\par

\subsubsection{Morty Seinfeld}

\par
Morty Seinfeld embodies the ``provider'' or ``hustler'' archetype. His
dialogue is disproportionately oriented toward \emph{Money} (15\%
of lines), while that topic hardly registers for other characters
(see Figure \ref{fig:topic_distribution}). The TF-IDF analysis
further emphasizes this with terms like ``dollar,'' ``pay,'' and
``wallet''. This quantitative profile correlates well with his
narrative function as a character always obsessed with deals, prices,
and financial survival. The spike in \emph{Money} dialogue during
Season 6 to almost 40\% of lines suggests entire narrative arcs were
constructed around this trait, reducing him to a unidimensional
figure to serve the plot.

\subsubsection{Frank Costanza}

\par
By contrast, Frank Costanza is an agent of chaos. The fact
that he leads the \emph{Emotion/Feeling} category by a wide margin
(approx. 25\%) confirms that he is a volatile character.
Incidentally, the \emph{Health} category for Frank features the term
``serenity'' (TF-IDF score 0.017). This lexical marker ironically links
his health issues with his emotional outbursts (the ``Serenity Now''
mantra), demonstrating how the show embeds catchphrases into
topically relevant speeches.

\subsubsection{Newman}

\par
Newman functions as a sort of ``hedonistic antagonist.''
Newman is the only character whose most dominant topics include both
\emph{Work} and \emph{Food}. The strong distinctiveness of
food-related terms in our corpus (``pie,'' ``cook,'' ``taste'') is often
related to Newman's scenes, as if his villainy is usually encoded by
gluttony or physical indulgence. His strong activity with \emph{Work}
dialogue is also remarkable; as the main cast generally avoids
talking of work, Newman often discusses his job (the post office),
frequently to complain or wield its bureaucratic power.

\subsection{Narrative Roles: Drivers vs. Reactors}

\par
The difference between ``drivers'' of plot and ``reactors'' is epitomized
in the difference between \textbf{Susan Ross} and \textbf{Helen Seinfeld}.

\subsubsection{Susan Ross}

\par
The profile for Susan is heavily weighted toward
\emph{Relationships}, effectively stripping her of an independent
thematic identity outside of her engagement with George Costanza.
Unlike Morty (Money) or Newman (Food), Susan does not have a ``hobby''
topic; her dialogue exists almost exclusively to advance the romantic
(and later tragic) plotline \cite{seinfeldwiki}. This is visually
confirmed by her evolution chart, where the shift from \emph{Relationships} to
\emph{Emotion} tracks the dissolution of her engagement and eventual
death, leaving no ``Miscellaneous'' or ``Culture'' buffer that would
indicate a fleshed-out persona.

\subsubsection{Helen Seinfeld}

\par
By contrast, Helen Seinfeld performs
the function of the ``narrative stabilizer.'' Her value is roughly
equally divided among Lifestyle, Relationships, and Emotion. Her
distribution lacks the sharp spikes of Frank or Morty, which
indicates her comic role is to represent a response to extremes,
rather than create new chaotic forces. She represents the "straight
woman" required to anchor the absurdities of the Seinfeld/Costanza dynamic.

\subsection{The Self-Referential Nature of ``Culture''}

\par
Finally, our TF-IDF analysis of the \emph{Culture} topic shows the
insular nature of the show's universe. The highest-scoring terms are
not general pop culture references but specific, self-referential
plot points like ``festivus'' and ``danson''. This indicates that when side
characters discuss ``culture,'' they are rarely discussing the outside
world; rather, they are discussing the specific, bizarre cultural
reality constructed within the show's writing. This reinforced the
``no hugging, no learning'' ethos by keeping the characters locked
within their own hermetic, self-obsessed loop.

% ===============================
% CONTRIBUTIONS
% ===============================
\section{Group Member contributions}

\par
All group members contributed equally to this project. Daniel Greco
focused primarily on data collection and statistical analysis.
Denis Tsariov and Kejun Fang concentrated on result
interpretation and thematic analysis, providing insights and
contextual understanding of the findings. Throughout the process, all
members collaborated on drafting, revising, and finalizing the
report, ensuring a cohesive and high-quality presentation of the work.

% ===============================
% REFERENCES
% ===============================

\renewcommand*{\bibfont}{\footnotesize}
\printbibliography

\end{document}
